\documentclass{casconpaper}
\title{\Large\sffamily{\bfseries{Towards Understanding Digital Information Discovery and Curation}}}
\author{
	Elena Voyloshnikova\\
	elenavoy@uvic.ca\\
	\and
	Dr. Margaret-Anne Storey\\
	mstorey@uvic.ca
}
\date{
	University of Victoria\\
	Victoria, BC, Canada\vspace{5ex}
}

\begin{document}

\maketitle
\thispagestyle{empty} % remove number on first page 

{\section*{Abstract\let\thefootnote\relax\footnotetext{Copyright \copyright\ 2014 Dr. Margaret-Anne Storey and Elena Voyloshnikova. Permission to copy is hereby granted provided the original copyright notice is reproduced in copies made.}}

Everyday life revolves around the discovery and curation of digital information. People search the Web continuously, from quickly looking up the information needed to complete a task, to endlessly searching for inspiration and knowledge. A variety of studies have modelled information seeking strategies and characterized information seeking and curation activities on the Web. However, there is a lack of research on how existing Web applications support the discovery and management of information, especially concerning the motivations behind them and how different approaches can be compared.

In this paper, we present a study of information discovery tools and how they relate to the nature of information seeking. We propose a conceptual framework that deals with the opportunistic and purposeful aspects of how people discover and manage digital information. This framework can be used when designing new, evaluating, or updating existing Web applications to match expanded use case objectives.


} % end abstract

{\section{Introduction}
Today, people commonly use Web technologies to satisfy their information needs. For example, some people research their travel destinations using various online resources, some shoppers look for product characteristics to make purchasing decisions online, and some look up precise locations when needed. In order to accommodate diverse and evolving user needs, web applications continuously introduce new features and services empowering information discovery and curation. 

Sometimes, Web users hope to find particular pieces of information, such as showtimes and phone numbers, to satisfy specific information needs. Other times, users lack well-articulated information needs, which results in their seeking behavior and knowledge gaps continuously changing. Often, people discover information online without even looking for it. The nature of information discovery can vary, and therefore, different types of it can require different tool support.  

In addition, people perform information curation tasks to manage, preserve, and add value to collections of information. With rapidly increasing popularity of socially curated information spaces, it is important to understand how to enable curation activities when designing tools to support information discovery.

To close their knowledge gaps, people turn to various Web technologies ranging from specialized search tools to visual discovery applications. Several studies have been directed at exploring high-level Web tasks, including information seeking tasks \cite{morrison, kellar2006, kellar2007}, deriving models of information seeking behaviors \cite{choo}, and looking at methods of information curation \cite{wittaker}. However, more research is necessary to determine how different tools and their features provide fundamental support for information discovery and curation.

To enhance information seeking and curating experiences and support users' interactions, we extend existing research by (1) deriving factors that enable information discovery and curation and relating them within a framework, (2) validating the framework by studying and describing currently existing web applications, and (3) providing questions to ask to address concepts of the framework when evaluating and designing new applications. Specifically, our research goal is to gain understanding of how existing tools support digital information curation and discovery. 

The remainder of this paper is organized as follows. Section 2 highlights some of the studies and technologies related to information curation and seeking tasks and behavior. Section 3 outlines the conceptual framework of factors and question that enable digital information discovery and support curation. In Section 4, we describe the methodology for validating the framework, and in Section 5 we  report sample design strategies that address the concepts in today's Web applications. Section 6 outlines the findings and relates them back to the literature. In Section 7, we describe the final framework, followed by future work and conclusion in sections 7 and 8 respectively.

} % end section


{\section{Related Work}

Several researchers have studied various characteristics of web usage behaviour. To gain understanding of how currently existing web tools support goal-oriented opportunistic web use, we study known mechanics of web usage, including high-level web tasks, information seeking behaviour, information curation, and modes of web use. 

Kellar et al. \cite{kellar2006} separated web tasks into five categories: transactions, browsing, fact finding, information gathering, and other uncategorized tasks.  In their later work, Kellar et al [ac] added communication and maintenance as additional web tasks. Similarly to Kellar, Sellen [ac] identified six web tasks that are commonly performed by web users: browsing, finding, housekeeping, information gathering, communicating, and transacting. Although Kellar et al. and Sellen make a clear distinction between different types of high-level tasks, it is evident that some information seeking tasks can overlap. For example, when gathering information about a new fitness studio, the user might be interested in its location. Location lookup is classified as fact finding; however, it contributes to the information gathering process and can be performed while the user is browsing.

At first, it seems convenient to think of user motivation behind various web tasks being conducting the task itself, such as communicating for the sake of communication. However, tasks such as browsing or fact funding can be motivated by higher goals, such as planning a weekend getaway or researching fitness programs and related topics. Therefore, when providing tool support for various tasks, it is useful to consider the motivation behind these tasks, as it may differ. Morrison et al. [ca] makes a distinction between methods of web use and purposes. The authors derived a purpose taxonomy web use, including three purposes or motivations: finding information, comparing or choosing to make a decision, and using the web to find relevant information to gain understanding of some subject. Consequently, methods of finding information identified by Morrison et al. are collecting, finding, exploring, and monitoring. The differences between the two taxonomies suggest that different information seeking tasks may be performed to satisfy each of the information seeking purposes. Therefore, each purpose may require more than one task-supporting mechanisms. 

A number of researchers have studies information seeking behaviour [ac]. Ellis et al [ac], proposed a model of information seeking characterized by six different patterns: starting, chaining, browsing, extracting, monitoring, and differentiating. Based on these patterns, Choo et al. [ac], derived corresponding anticipated web moves. According to authors [ac], when users identify sources of interest, they usually identify which web sites can point to that information of interest.  Chaining occurs when users navigate through links on those initial pages. When people browse, they scan top-level pages, headings, lists, and site maps. Differentiating takes place when people bookmark, print, copy and paste information, or choose earlier selected site. Monitoring occurs when users revisit web pages or receive updates from some earlier visited sites. Finally, extracting can occur when the user systematically searches the site to extract information of interest. 

In [insert year], Bates [ac] proposed a model of four information seeking modes that consists of being aware, monitoring, browsing, and searching. Bates differentiated the modes based on the levels of attention being active or passive, and information needs being directed or undirected. Thus, browsing can be characterized as undirected active information seeking because users do not know directly what information they are looking for, but they actively look for some information. Searching falls under active directed information seeking because the information need is clearly defined s the search is directed. Finally, monitoring and being aware are passive modes of information seeking although monitoring is directed, and being aware is undirected.   

In [insert year], Bates [ac] extended her research with the notion of information farming. Information farming involves people collect and organize information for future use and revisitation. More commonly, it is referred to as digital curation, which is the notion of collecting and managing digital information for the purpose of adding value to the collection, and revisitation [ac]. Wittaker [ac] believes that in terms of web use, there is happening a significant shift from information consumption to information curation, which means that people no longer just use the web to find and consume the information that they are interested in, including the information related to their personal endeavors, but they also try to save and manage that information so that later it can be reaccessed and exploited, at times for the purpose of goal realization. 

Categorizing web usage into information seeking,  digital curation, and other web tasks does not necessarily give full insight about how web usage related to personal goals is performed. Lindley et al. [ac] conducted a qualitative study involving 24 participants tracking their daily web usages in a form of a diary. As a result of this study, the researchers identified five distinct modes of web use: respite, orienting, opportunistic, purposeful, and lean-back [ac]. According to authors [ac], people web browse opportunistically when they look for information related to some personal interest, long-term goal, or future ambition. Purposeful use occurs when the user knows for a fact what information she needs to acquire or what action online she needs to perform in order to continue or finish some other activity. Respite mode usually occurs when users are in the process of waiting for something or taking a break, and it serves as means for people to temporarily occupy themselves when high engagement with the content is not a requirement. Orienting mode usually occurs when people want to be updated on what has been happening in their environment. Examples of this mode are checking email at work or looking at the news and updates on a social networking site. Finally, lean-back mode of web use can be thought of as listening to the radio or watching TV. It usually involves watching videos online or browsing through other types of entertainment content. 

Lindley’s et al. [ac] primary motivations behind looking at use modes that occur when people browse Internet was that traditional web use studies and web tasks discovered by other researchers could not reflect the depth of user’s intentions online. It is evident that modes of web use make user’s motivations clearer. In the context of goal-oriented activities, the mode that is best associated with user’s long-term ambitions and personal hobbies and interests is opportunistic. Purposeful use often relates to immediate goals that feed into some wider context. 

To further compare opportunistic and purposeful use modes, we summarized some essential differences between the two in Table 1. Opportunistic mode is usually performed because it relates to some user interest  and the act of performing it is enjoyable by itself, whereas purposeful mode usually serves as a tool for completion or continuation of some other activity. An information need for opportunistic use can be blur and change continuously throughout web use session, whereas with purposeful use, people know exactly what information they are interested in and that information need usually does not change. Purposeful use has a well-defined point of completion, which occurs when objective is achieved or information is acquired, whereas with opportunistic use, people often cannot indicate completion of the web task and they finish whenever they have been browsing Internet for too long or they need to complete some other task of higher priority. Finally, opportunistic use is ‘grasshopper-like’: users jump from one resource to another. Purposeful use is characterized by direct actions or search. Motivation, information need, and duration, and usage patterns are the four characteristics that we will use in order to estimate whether a task is performed during the opportunistic mode of web use.

A number of researchers have studied how people use the web, including major tasks, information seeking behaviours, and information curation. However, there is lack of research on how currently existing technologies support people in performing these tasks in the context of goal-associated user intentions. With this overview of related work in mind, we proceed with building a framework of web application design elements for goal-oriented opportunistic web use.
} % end section




{\section{Conceptual Framework}
\begin{table*}[htbp]
\caption{Conceptual Framework.}
\centering
\begin{tabular}{|p{0.30\linewidth}|p{0.70\linewidth}|}
\hline
&\\
\textbf{\large{Factors}}   & \textbf{\large{Questions}}  \\
\hline
&\\
\textbf{\large{Discovery}}                     &                                                                                                           \\

&\\
\emph{\textbf{Serendipitous discovery}}     &                                                                                                           \\

Arbitrary navigation         & Does the application provide means for arbitrary navigation among resources?                              \\
Search-based navigation      & Does the search feature help discover arbitrary resources related to the topic of interest?               \\
Navigation within a directory & Do directory categories suggest and help navigating to resources related to the topic of interest?           \\
Integration                  & Does the application support navigation to information sources beyond the application?                   \\
Visual preview               & Do resources have visual previews?                                                                        \\
Spatial arrangement          & Are resources presented in a spatially meaningful way?                                                    \\
&\\
\emph{\textbf{Fact discovery}}                &                                                                                                           \\
Search-based navigation      & Does the search feature help discover the specific resource of interest?                                  \\
Navigation within a directory & Do directory categories help narrow results to specific types of resources?                                   \\
Uniform representation       & Are resources presented in a uniform way?                                                                 \\
&\\
\emph{\textbf{Rediscovery}}                     &                                                                                                           \\
History-based rediscovery    & Does the application save and provide access to browsing history?                                        \\
Bookmark-based rediscovery   & Does the application support bookmark-based resource revisitation?                                        \\
Search-based rediscovery     & Is the search within application a reliable method for resource revisitation?                             \\
&\\
\emph{\textbf{Subscription-based discovery}}          &                                                                                                           \\
Site subscription            & Does the application allow subscriptions to news and updates?                                             \\
User subscription             & Does the application allow subscriptions to other users' activities?                                      \\
Notifications                & Does the application have notification mechanism(s)?                                                      \\
News stream                  & Can subscription updates be visible within the application?  \\
&\\
\hline     
&\\                                        
\textbf{\large{Curation}}                     &                                                                                                        \\
&\\        
\emph{\textbf{Management}}                    &                                                                                                           \\
Categorization               & Does the application support information categorization?                                                  \\
&\\
\emph{\textbf{Preservation}}                   &                                                                                                           \\
Internal Preservation        & Does the application have bookmarking mechanism for preserving information within the application?        \\
External Preservation        & Does the application have bookmarking mechanism(s) for preserving information outside of the application? \\
&\\
\emph{\textbf{Information enhancement}}            &                                                                                                           \\
Evaluation                   & Can resources be evaluated?                                                                               \\
Annotation                   & Can resources be annotated?                                                                               \\
&\\
\emph{\textbf{Social curation}}              &                                                                                                           \\
Adding resources             & Can resources be added to the publicly available pool of information from outside of the application?     \\
Resharing resources          & Can resources be publicly reshared within the application?         \\     
&\\          
\hline
\end{tabular}
\end{table*}

Although information discovery and curation tasks elicit predominant portion of interactions within many Web applications today, there is lack of systematic guidelines for designing and evaluating applications that could afford these tasks. We hope to reduce this gap by developing a framework of design factors that enable digital information discovery and curation. 

The framework (see Table 1) consists of two main categories of factors, discovery and curation, that are consequently broken down into subcategories. Each subcategory contains factors that determine use case enablers and corresponding questions to ask when designing or evaluating an application. This section outlines main components of the framework.

} % end section

{\subsection{Information Discovery}
A number of researchers attempted classification of information seeking tasks and methods. In our framework, we built on existing research to derive corresponding design factors. Thus, the discovery category consists of serendipitous discovery, fact discovery, rediscovery, and channel-based discovery. 
} % end subsection

{\subsubsection{Serendipitous discovery}
Serendipitous discovery refers to information discovery resulting from serendipitous  browsing. Such discovery is characterized by underdefined, absent, or hidden information need, and it is usually involves browsing through diverse resources with varying content types \cite{kellar2006}. The following is the key criteria that influence serendipitous information discovery:

\begin{itemize}
  \item \textbf{Arbitrary navigation.} In order to browse through diverse information, an information discovery tool needs to provide a way to arbitrary navigate among resources adapting to continuously changing information needs and interests.
  \item \textbf{Search-based navigation.} Search-based navigation helps retrieve information sufficient for identifying an entry point of search \cite{levene}. In case of serendipitous discovery, since the information need is not well-articulated, the search feature should provide resources related to a broad topic of interest.
  \item \textbf{Navigation within a directory.} Similarly to search-based navigation, navigation within a directory should provide a way to narrow the results to those related to one topic. In addition, navigation with a dictionary can suggest topics of interest and help the user formulate an information need \cite{levene}.
  \item \textbf{Integration.} To users with ambiguous information needs, one information portal might not provide access to all information of interest. If an information discovery application gives access to resources from various sources, the user should be able to navigate to original sources.
  \item \textbf{Visual preview.} Abrams et al. \cite{abrams} identified link representation as one of the problems with traditional bookmarking. Analogously with browsing through a bookmark manager, identifying relevant information when browsing through links to diverse resources can be a challenging task. Visual preview should make it easier to evaluate relevance of resources.
  \item \textbf{Spatial arrangement.} Similarly to link representation, spatial visualization of numerous links is another problem that occurs when browsing through diverse content \cite{abrams}. Therefore, meaningful spatial arrangement is important when displaying links to diverse resources.
\end{itemize}


} % end subsubsection

{\subsubsection{Fact discovery}
Fact discovery refers to information discovery resulting from looking for a specific piece of information. It is characterized by a well-defined information need and is easier to perform within systems that provide access to homogeneous types of information \cite{kellar2006, lindley}. Below is the list of factors that influence fact discovery: 
\begin{itemize}
  \item \textbf{Search-based navigation.} Since with fact discovery an information need is known, the goal of search-based navigation is to navigate to the resource of interest as directly as possible. Therefore search should limit resources to make it possible to find the one resource of interest. 
  \item \textbf{Navigation within a directory.} Navigation within a directory is used to direct the user to relevant resources. In case of fact discovery, it should narrow the results to a specific type of resource so that further fact discovery is bounded by that type. 
  \item \textbf{Uniform representation.} Uniform representation is a method of displaying diverse resources in a common way, with each resource having the same set of components. Such representation assures that each resource has the same set of facts associated with it, and therefore, the user can afford to have expectations about information that she can find when looking for a specific fact.
\end{itemize}

} % end subsubsection

{\subsubsection{Rediscovery}
Rediscovery refers to to information discovery resulting from revisiting previously discovered resources. The following is a list of factors that enable rediscovery:
\begin{itemize}
  \item \textbf{History-based rediscovery.} A Web application needs to automatically backup browsing history in order to enable history-based rediscovery. History-based rediscovery can be useful when no bookmarking took place when the resource was discovered for the first time.  
  \item \textbf{Bookmark-based rediscovery.} Bookmark-based rediscovery is one of the most common ways of information revisitation. The majority of Web browsers are equipped with bookmarking mechanisms. However, some modern Web applications provide integrated mechanisms for bookmarking and bookmark-based information rediscovery within application itself.
  \item \textbf{Search-based rediscovery.} Search-based rediscovery is not always a reliable way of refinding information. In information portals that provide access to fairly ambiguous information and that have information regularly repopulated and updated,   search is usually designed around retrieving information related to some topic but not very specific. In order to revisit a resource, search must provide consistent results.
\end{itemize}
} % end subsubsection

{\subsubsection{Subscription-based discovery}
Subscription-based discovery is based on a combination of monitoring and awareness types of information seeking. It occurs when information is suggested to users based on their subscriptions. If users can actively look for updates, then an application affords monitoring. If users can receive notifications about updates, then an application facilitates awareness.                            

\begin{itemize}
  \item \textbf{Site subscription.} Subscriptions to updates from a site help users follow the news. In order to support subscription-based discovery, an application must provide a subscription mechanism.
  \item \textbf{User subscription.} Similarly to site subscriptions, user subscriptions  help following individual users and their activities. Such subscriptions help to further filter the content delivered to the user.
  \item \textbf{Notifications.} Notification mechanisms enable user awareness about the  new content on the subscribed channel. 
  \item \textbf{News stream.} Displaying news stream within application further promotes awareness and can serve as a monitoring mechanism.
\end{itemize}
} % end subsubsection

{\subsection{Information Curation}
By definition, digital information curation is the notion of managing, preserving, and adding value to collections of information. Thus, the curation category consists of information management, preservation, information enhancement, and social curation.  
} % end subsection

{\subsubsection{Management}
Information management is one of the key elements of information curation. In the context of Web information management, the following factor plays the major role:
\begin{itemize}
  \item \textbf{Categorization.} Resource categorization helps establish relationships between various resources. Allowing to customary categorize information can aid rediscovery, discovery in a socially curated space, as well add value to the resource.
\end{itemize}
} % end subsubsection

{\subsubsection{Information Preservation}
Information preservation, or information gathering, is a common Web tasks that is usually performed with an intent of information revisitation. However, in the case of opportunistic web use, information gathering is sometimes performed with just a goal of collecting information rather than revisiting it in the future. Traditional information preservation mechanism is bookmarking. Many Web applications apply variations  of bookmarking mechanisms. 
\begin{itemize}
  \item \textbf{Internal Preservation.} Internal information preservation means bookmarking resources within application to be accessible within the same application. Such bookmarking facilitates information curation within the system.
  
  \item \textbf{External Preservation.} External preservation means bookmarking resources to be available within some other bookmarking system. An application must facilitate integration with other applications in order to enable external preservation.
\end{itemize}
} % end subsubsection

{\subsubsection{Information Enhancement}
One of the important elements of digital curation is adding value to information overtime. In other words, enhancing already present information. Some forms of information enhancement are prevalent mainly in the context of social curation. Others can be performed in the context of personal curation. 
\begin{itemize}
  \item \textbf{Evaluation.} Evaluation methods vary from ratings to 'likes'. They usually take place in socially curated information systems. However, some of them contribute to personal reflection and information preservation. 
  \item \textbf{Annotation.} Annotations are metadata attached to a resource. They can take have forms of comments, tags, descriptions, etc. Annotations make it easier to search for resources as well as to interpret them. 
 
\end{itemize}
} % end subsubsection

{\subsubsection{Social Curation}
In addition to public resource sharing, social information curation encapsulates all of the elements of digital curation. Therefore, adding resources and resharing those resources are the two key factors that empower social curation.
\begin{itemize}
  \item \textbf{Adding resources.} Adding resources does not only facilitate global Web information curation but also scales the information available through the system up providing more opportunities for information discovery.
  \item \textbf{Resharing resources.} Resharing currently existing within the system resources supports subscription-based information discovery since it supports channeling information that can be of interest to subscribed groups of users. 
 
\end{itemize}
} % end subsubsection

The following section describes the methodology for evaluating the conceptual framework and understanding how to address the elements of the framework when designing real world applications.

{\section{Methodology}
The study presented in this paper has two primary goals. The first goal is to validate the conceptual framework of factors that enable digital information discovery and curation. The second goal of the study is to gain perspectives on how to address different elements of our framework when designing real-world applications. Therefore, our research questions are the following:
\\

\emph{RQ1: How do existing Web applications support information discovery?}

\emph{RQ2: How do existing Web applications support information curation?}\\


Our methodology for studying existing Web applications is based on Yin’s guidelines for designing a case study \cite{yin}. The motivation behind choosing a case study over other methods of qualitative research was based on our choice of research questions, which have an explanatory nature, lack of control over existing applications and their development, and having to focus on contemporary use of real-life web applications. According to Yin \cite{yin}, case study would be an optimal research strategy given above characteristics of the subject matter.

To answer our research questions, we analyzed twenty different cases. For each case, we picked a Web application whose primary purpose is to be used for information discovery. We examined overall purpose of each application, its description as defined by the site itself, literature and documentation related to the case, if they were available, against the features that the application provides. For example, if an application provides bookmarking features, we would check if it is indeed intended to be used for information preservation.

To increase external validity of our study, we chose the cases based on replication logic \cite{yin}. Using replication logic in a case study design means carefully selecting each case so that it either predicts analogous results or predicts contrasting results but for anticipated reasons. Therefore, we used our conceptual framework (see Section 3) to predict if an application supports each of the information discovery and curation tasks based on the features that the application provides. 

Consequently, our case study had a form of an iterative process of selecting cases, analyzing each case, and determining whether or not it meets our theoretical propositions. If one or more cases did not support the theory, then we modified the propositions and selected a new set of cases until the results of analyzing the case gave the anticipated results for all cases. We then grouped features into factors that enable information discovery and curation, and recorded corresponding questions to ask in order to evaluate an application. The resulting framework is depicted in Table 1. Limitations of our study are presented in Section 6.
} % end section


{\section{Digital Information Discovery and Curation}

Through examining twenty different Web applications whose main function is to help people discover information, we were able to build and validate the conceptual framework of factors and questions that facilitate digital information discovery and curation. In this section, we present our findings and answer the research questions, RQ1 and RQ2. \\

\emph{RQ1: How do existing Web applications support information discovery?} \\

There is a large spectrum of usages within information discovery systems.  

Applications that facilitate serendipitous information discovery often employ elaborate resource representation techniques. Social-bookmarking systems, such as Pinterest, Scoop.it!, and StumbleUpon, had indeed tried solving the two link representation problems indicated by Abrams \cite{abrams}, visualization and representation.  

Search-based rediscovery can be a challenging task within some applications and a very simple in others.  \\

\emph{RQ2: How do existing Web applications support information curation?}\\
 
} % end subsection





{\section{Limitations and Threats to Validity}

} % end section

{\section{ Future Work}
One of the possible future research objectives would be to test the framework on a real-world application, and to either enhance its use as an opportunistic goal-oriented application, or to extend its use  to support opportunistic information discovery, information curation, or goal realization. 

} % end section

{\section{Conclusion}
 
} % end section



{\begin{thebibliography}{9}
\bibitem{abrams} Abrams, David, Ron Baecker, and Mark Chignell. "Information archiving with bookmarks: personal Web space construction and organization." \emph{Proceedings of the SIGCHI conference on Human factors in computing systems}. ACM Press/Addison-Wesley Publishing Co., 1998.


\bibitem{beagrie}Beagrie, Neil. "Digital curation for science, digital libraries, and individuals." \emph{International Journal of Digital Curation} 1.1 (2008): 3-16.




\bibitem{choo}Choo, C. W., Detlor, B., and Tunbull, D. (2000). Information seeking on the web: An integrated model of browsing and searching.  \emph{FirstMonday}, 5(2). Available from http://firstmonday.org/issues/issue5\_
2/choo/index.html.

\bibitem{foster}Foster, Allen, and Nigel Ford. "Serendipity and information seeking: an empirical study." \emph{Journal of Documentation} 59.3 (2003): 321-340.

\bibitem{subscr}
	Java, Akshay, et al. "Feeds That Matter: A Study of Bloglines Subscriptions." \emph{ ICWSM.} 2007.
   
\bibitem{kellar2006} Kellar, Melanie, Carolyn Watters, and Michael Shepherd. "A Goal-based Classification of Web Information Tasks." \emph{Proceedings of the American Society for Information Science and Technology} 43.1 (2006): 1-22.

\bibitem{kellar2007}Kellar, Melanie, Carolyn Watters, and Michael Shepherd. "A field study characterizing Web-based information-seeking tasks." \emph{Journal of the American Society for Information Science and Technology} 58.7 (2007): 999-1018.

\bibitem{levene}Levene, Mark.  \emph{An introduction to search engines and web navigation.} John Wiley \& Sons, 2011.

\bibitem{lindley}Lindley, Siân E., et al. "It's simply integral to what I do: enquiries into how the web is weaved into everyday life." \emph{Proceedings of the 21st international conference on World Wide Web.} ACM, 2012.

\bibitem{mishne}Mishne, Gilad, and Maarten De Rijke. "A study of blog search." \emph{Advances in information retrieval.} Springer Berlin Heidelberg, 2006. 289-301.

\bibitem{morrison}Morrison, Julie B., Peter Pirolli, and Stuart K. Card. "A taxonomic analysis of what World Wide Web activities significantly impact people's decisions and actions." \emph{CHI'01 extended abstracts on Human factors in computing systems.} ACM, 2001.

\bibitem{wittaker}Whittaker, Steve. "Personal information management: from information consumption to curation." \emph{Annual review of information science and technology} 45.1 (2011): 1-62.

\bibitem{yin} Yin, R. K. 2009. \emph{Case study research}, 4th, Thousand Oaks, CA: Sage.
    
\end{thebibliography}
} % end references

\end{document}
